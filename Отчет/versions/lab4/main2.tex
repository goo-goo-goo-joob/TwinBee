%!TEX TS-program = xelatex

% Шаблон документа LaTeX создан в 2018 году
% Алексеем Подчезерцевым
% В качестве исходных использованы шаблоны
% 	Данилом Фёдоровых (danil@fedorovykh.ru) 
%		https://www.writelatex.com/coursera/latex/5.2.2
%	LaTeX-шаблон для русской кандидатской диссертации и её автореферата.
%		https://github.com/AndreyAkinshin/Russian-Phd-LaTeX-Dissertation-Template

\documentclass[a4paper,14pt]{article}

\input{data/preambular.tex}
\begin{document} % конец преамбулы, начало документа
\begin{titlepage}
	\begin{center}
		ФЕДЕРАЛЬНОЕ  ГОСУДАРСТВЕННОЕ АВТОНОМНОЕ \\
		ОБРАЗОВАТЕЛЬНОЕ УЧРЕЖДЕНИЕ ВЫСШЕГО ОБРАЗОВАНИЯ\\
		«НАЦИОНАЛЬНЫЙ ИССЛЕДОВАТЕЛЬСКИЙ УНИВЕРСИТЕТ\\
		«ВЫСШАЯ ШКОЛА ЭКОНОМИКИ»
	\end{center}
	
	\begin{center}
		\textbf{Московский институт электроники и математики}
		
		\textbf{Им. А.Н.Тихонова НИУ ВШЭ}
	\end{center}
	\vspace{1ex}	
	\begin{center}
		Солодянкин Андрей Александрович, группа БИВ172\\
		Подчезерцев Алексей Евгеньевич, группа БИВ172
	\end{center}	
	\vspace{1ex}
	\begin{center}
		\textbf{ОТЧЕТ\\
			ПО ЛАБОРАТОРНОЙ РАБОТЕ 3
		}
	\end{center}	
	\vspace{2ex}
	\begin{center}
		по дисциплине «Электроника»\\
		Тема: «Переходные процессы в RC- и RL- цепях»
	\end{center}
	\vspace{2ex}
	\begin{center}
		Вариант 1.\\
		Дата сдачи отчета: \today
	\end{center}
	\vspace{2ex}
	\vfill
	\begin{center}
		Москва \the\year г.
	\end{center}
\end{titlepage}
\setcounter{page}{2} % нумерация

\renewcommand\contentsname{\centering {\normalsize Содержание}}
\tableofcontents
\newpage

\section*{Постановка задачи}
\addcontentsline{toc}{section}{Постановка задачи}

Реализовать загрузку растровых изображений и переработать функционал отрисовки объектов таким образом, чтобы для отображения объектов использовались изображения. Окончательно доработать логику игры.


\newpage

\section{Основная часть}
\subsection{Общая идея решения задачи}
В работе была реализована отрисовка объектов с помощью растровых изображений вместо геометрических фигур, используя класс \textit{QImage}. Была завершена разработка логики игры: появилась возможность отображать счет и количество жизней на экране, обработка столкновений пчелы и врагов. Был улучшен механизм взаимодействий объектов.

\subsection{Структура и принципы действия}
В классе конкретного посетителя \textit{DrawGameItems} был изменен механизм отрисовки объектов (функция \textit{visit}). С помощью класса \textit{QImage} и его методов \textit{load} и \textit{scaled} было загружено растровое изображение, которое масштабировалось под заданные размеры (в классе \textit{GameItem} появились поля \textit{sizeX} и \textit{sizeY} для хранения ширины и высоты объекта). Таким образом, объекты классов \textit{Bee}, \textit{Cloud}, \textit{Bell}, \textit{RedEnemy} и \textit{BlueEnemy} изображаются на экране с помощью выбранных картинок. Также был добавлен задний фон и установлены необходимые шрифты для отображения текста.

В классе \textit{Game} была улучшена функция \textit{Collide}, обрабатывающая столкновения объектов: стали использоваться поля \textit{sizeX} и \textit{sizeY} для расчета столкновений, была добавлена обработка взаимодействия врагов и пчелы; при их столкновении у пчелы отнимается жизнь, при этом количество жизней не может быть отрицательным? и потеря жизни не может происходить чаще заданного времени \textit{lifeTime}. Данный алгоритм описан в классе \textit{Bee} в функциях \textit{Update} и \textit{SubLifes}.

Был доработан механизм окончания игры: когда количество жизней достигает 0, то игра останавливается (перестает производиться обработка нажатий и отрисовка), выводится надпись \textit{"Game over"}.

\begin{figure}[H]
	\centering
	\includegraphics[width=\linewidth]{ClassDiagram1.png}
	\caption{Диграма классов}	
\end{figure}
\begin{figure}[H]
	\centering
	\includegraphics[width=\linewidth]{ClassDiagram2.png}
	\caption{Диграма классов (продолжение)}	
\end{figure}

\subsection{Процедура получения исполняемых программных модулей}
Программный код был скомпилирован с среде \textit{Qt Creator}. Компиляция раздельная: исходный код программы разделён на несколько файлов. Никаких дополнительных ключей не добавлялось, использовались ключи, которые добавляются по умолчанию. Параметры сборки: компилятор С++ MinGW 4.8.2, профиль Qt: Qt 4.8.7, отладчик GDB.
\subsection{Результаты тестирования}
Тестирование программы представлено в файле \textit{"main.cpp"} в функции \textit{Main()}. Ожидаемая отрисовка объектов в окне \textit{MainWindow}:

\begin{figure}[H]
	\centering
	\includegraphics[width=\linewidth]{test.jpg}
	\caption{Результат тестирования}	
\end{figure}

\newpage
\setcounter{figure}{1} 
\setcounter{section}{1} 
\setcounter{subsection}{1} 

\begin{center}
	\section*{Приложение А}
	код программы (изменения)
	\addcontentsline{toc}{section}{Приложение А}
\end{center}
\renewcommand{\subsection}{\Asbuk{section}.\arabic{subsection}}
\setcounter{subsection}{1} 
\textbf{\subsection{  - MainWindow.cpp}}
\addcontentsline{toc}{subsection}{MainWindow.cpp}
\begin{lstlisting}
MainWindow::MainWindow(QWidget *parent) :
	QMainWindow(parent),
	ui(new Ui::MainWindow)
{
	ui->setupUi(this);
	Game& game = Game::Instance();
	this->setFixedSize(game.width(),game.height());
	Notifer::Instance().Subscribe(this);
	QPixmap bkgnd("images/back.png");
	bkgnd = bkgnd.scaled(this->size(), Qt::IgnoreAspectRatio);
	QPalette palette;
	palette.setBrush(QPalette::Background, bkgnd);
	this->setPalette(palette);
	QFont f( "Arial", 15, QFont::Bold);

	score = new QLabel(this);
	score->setFrameStyle(QFrame::Sunken);
	score->setGeometry(QRect(10,30,200,20));
	score->setStyleSheet("QLabel {color : white; }");
	score->setFont( f);
	lifes = new QLabel(this);
	lifes->setFrameStyle(QFrame::Sunken);
	lifes->setGeometry(QRect(10,10,200,20));
	lifes->setStyleSheet("QLabel {color : white; }");
	lifes->setFont( f);

	QFont f1( "Arial", 30, QFont::Bold);
	over = new QLabel(this);
	over->setFrameStyle(QFrame::Sunken);
	over->setGeometry(QRect(game.width()*0.36,game.height()/2,220,30));
	over->setStyleSheet("QLabel {color : white; }");
	over->setFont( f1);
};
\end{lstlisting}

\setcounter{subsection}{2} 
\textbf{\subsection{  - MainWindow.h}}
\addcontentsline{toc}{subsection}{MainWindow.h}
\lstinputlisting[language=C++]{versions/lab4/code/mainwindow.h}

\setcounter{subsection}{3} 
\textbf{\subsection{  - GameItem.cpp}}
\addcontentsline{toc}{subsection}{GameItem.cpp}
\lstinputlisting[language=C++]{versions/lab4/code/GameItem.cpp}

\setcounter{subsection}{4} 
\textbf{\subsection{  - Game.h}}
\addcontentsline{toc}{subsection}{Game.h}
\lstinputlisting[language=C++]{versions/lab4/code/Game.h}

\setcounter{subsection}{5} 
\textbf{\subsection{  - Bee.h}}
\addcontentsline{toc}{subsection}{Bee.h}
\lstinputlisting[language=C++]{versions/lab4/code/Bee.h}

\setcounter{subsection}{6} 
\textbf{\subsection{  - Bee.cpp}}
\addcontentsline{toc}{subsection}{Bee.cpp}
\lstinputlisting[language=C++]{versions/lab4/code/Bee.cpp}
\end{document} % конец документа
