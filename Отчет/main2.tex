%!TEX TS-program = xelatex

% Шаблон документа LaTeX создан в 2018 году
% Алексеем Подчезерцевым
% В качестве исходных использованы шаблоны
% 	Данилом Фёдоровых (danil@fedorovykh.ru) 
%		https://www.writelatex.com/coursera/latex/5.2.2
%	LaTeX-шаблон для русской кандидатской диссертации и её автореферата.
%		https://github.com/AndreyAkinshin/Russian-Phd-LaTeX-Dissertation-Template

\documentclass[a4paper,14pt]{article}

\input{data/preambular.tex}
\begin{document} % конец преамбулы, начало документа
\begin{titlepage}
	\begin{center}
		ФЕДЕРАЛЬНОЕ  ГОСУДАРСТВЕННОЕ АВТОНОМНОЕ \\
		ОБРАЗОВАТЕЛЬНОЕ УЧРЕЖДЕНИЕ ВЫСШЕГО ОБРАЗОВАНИЯ\\
		«НАЦИОНАЛЬНЫЙ ИССЛЕДОВАТЕЛЬСКИЙ УНИВЕРСИТЕТ\\
		«ВЫСШАЯ ШКОЛА ЭКОНОМИКИ»
	\end{center}
	
	\begin{center}
		\textbf{Московский институт электроники и математики}
		
		\textbf{Им. А.Н.Тихонова НИУ ВШЭ}
	\end{center}
	\vspace{1ex}	
	\begin{center}
		Солодянкин Андрей Александрович, группа БИВ172\\
		Подчезерцев Алексей Евгеньевич, группа БИВ172
	\end{center}	
	\vspace{1ex}
	\begin{center}
		\textbf{ОТЧЕТ\\
			ПО ЛАБОРАТОРНОЙ РАБОТЕ 3
		}
	\end{center}	
	\vspace{2ex}
	\begin{center}
		по дисциплине «Электроника»\\
		Тема: «Переходные процессы в RC- и RL- цепях»
	\end{center}
	\vspace{2ex}
	\begin{center}
		Вариант 1.\\
		Дата сдачи отчета: \today
	\end{center}
	\vspace{2ex}
	\vfill
	\begin{center}
		Москва \the\year г.
	\end{center}
\end{titlepage}
\setcounter{page}{2} % нумерация

\renewcommand\contentsname{\centering {\normalsize Содержание}}
\tableofcontents
\newpage

\section*{Постановка задачи}
\addcontentsline{toc}{section}{Постановка задачи}

Разработать и реализовать механизм таймера с использованием класса \textit{QTimer} библиотеки \textit{Qt}. Разработать и реализовать классы для создания меню приложения. Доработать полностью функционал игры. Реализовать возможность сохранения игры и последующего ее продолжения. Задействовать минимум два паттерна проектирования. Приложение должно содержать минимум два потока выполнения: использовать средства языка Си++ либо класс \textit{QThread} библиотеки \textit{Qt}. Результатом данной лабораторной работы должно являться полноценное графическое приложение.



\newpage

\section{Основная часть}
\subsection{Общая идея решения задачи}
В данной работе для создания меню приложения были использованы средства библиотеки \textit{Qt}: классы \textit{QMenu} и \textit{QAction}. Был полностью доработан функционал игры: появилась возможность сохранения игры, продолжения последней сохраненной игры; возможность случайной генерации облаков и генерации врагов с помощью файла; появились переключения между уровнями, отработан механизм проигрыша и запуска новой игры. Функция обработки столкновений была вынесена в отдельный поток.

Для выполнения задачи был использован механизм таймера, реализованный в 3 лабораторной работе. Также в 3 лабораторной работе были реализованы паттерны проектирования: адаптер и одиночка.

\subsection{Структура и принципы действия}
В класс \textit{MainWindow} были добавлены поля \textit{QMenu} и \textit{QAction}, а в конструкторе класса был применен метод \textit{connect} для подключения сигнала нажатия кнопки из меню к необходимым слотам, которые также были добавлены в класс \textit{MainWindow}. В меню находится кнопка \textit{Help} (содержит информацию об игре), \textit{Save game} (сохраняет текущую игру) и \textit{New game} (начинает игру сначала). Также для элементов меню \textit{Save game} и \textit{New game} реализован механизм горячих клавиш. 

Возможность сохранения игры была реализована с помощью добавления дополнительного \textit{ini}-файла (\textit{"settings\_saved.ini"}) с записанными во время игры настройками. При запуске игры вызывется окно с сообщением (используется класс \textit{QMessageBox}), если существует файл с сохраненными настройками. С помощью стандартного механизма кнопок \textit{Ok} и \textit{Cancel} пользователю предоставляется возможность начать новую игру или загрузить последнюю сохраненную. Класс Initialization был незначительно изменен для реализации возможности записи в файл "\textit{settings\_saved.ini}".

Возможность случайной генерации облаков была добавлена в функцию \textit{Update} класса \textit{Game}, используя класс \textit{Notifer}, в котором был разработан механизм таймера. Методы \textit{setLevel1} и \textit{setLevel2} позволяют считывать из файла информацию о типе и количестве врагов, которых необходимо будет сгенерировать. Функции записывают считанную информацию в очередь \textit{box}. Также был написан новый метод класса \textit{Game} \textit{fromTheBox}: он позволяет брать информацию из поля \textit{box} и генерировать необходимое количество врагов заданного типа в соответствии с механизмом таймера. Когда очередь \textit{box} для уровня 1 опустошается, то происходит переключение между уровнями, выводится необходимая надпись и запускается метод \textit{setLevel2}. Когда количество жизней достигает 0, то игра останавливается и выводится соответствующая надпись.

Для вынесения функции обработки столкновений в отдельный поток был реализован класс \textit{Worker} (наследуется от \textit{QObject}) c механизмом слотов и сигналов: слот \textit{doWork} непосредственно обрабатывает столкновения, после выполнения слот выбрасывает сигнал \textit{resultReady}. В самом классе \textit{Game} используется экземпляр класса \textit{QThread} \textit{workerThread}, который запускается в конструкторе и останавливается в деструкторе. Класс \textit{Game} также наследуется от \textit{QObject}, что позволяет добавить механизм слотов (\textit{handleResults}) и сигналов (\textit{operate}). Теперь вместо функции \textit{collide} будет вызыветься сигнал \textit{operate} (с помощью функции \textit{connect} соединен со слотом класса \textit{Worker}, а сигнал \textit{resultReady} - со слотом \textit{handleResults}).

\begin{figure}[H]
	\centering
	\includegraphics[width=\linewidth]{ClassDiagram1.png}
	\caption{Диграма классов}	
\end{figure}
\begin{figure}[H]
	\centering
	\includegraphics[width=\linewidth]{ClassDiagram2.png}
	\caption{Диграма классов (продолжение)}	
\end{figure}

\subsection{Процедура получения исполняемых программных модулей}
Программный код был скомпилирован с среде \textit{Qt Creator}. Компиляция раздельная: исходный код программы разделён на несколько файлов. Никаких дополнительных ключей не добавлялось, использовались ключи, которые добавляются по умолчанию. Параметры сборки: компилятор С++ MinGW 4.8.2, профиль Qt: Qt 4.8.7, отладчик GDB.
\subsection{Результаты тестирования}
Тестирование программы представлено в файле \textit{"main.cpp"} в функции \textit{Main()}. Ожидаемая отрисовка объектов:

\begin{figure}[H]
	\centering
	\includegraphics[]{test1.jpg}
	\includegraphics[width=\linewidth]{test2.jpg}
	\caption{Результаты тестирования}	
\end{figure}

\newpage
\setcounter{figure}{1} 
\setcounter{section}{1} 
\setcounter{subsection}{1} 

\begin{center}
	\section*{Приложение А}
	код программы (изменения)
	\addcontentsline{toc}{section}{Приложение А}
\end{center}
\renewcommand{\subsection}{\Asbuk{section}.\arabic{subsection}}
\setcounter{subsection}{1} 
\textbf{\subsection{  - Game.h}}
\addcontentsline{toc}{subsection}{Game.h}
\lstinputlisting[language=C++]{../QtGame/Game.h}

\setcounter{subsection}{2} 
\textbf{\subsection{  - Game.cpp}}
\addcontentsline{toc}{subsection}{Game.cpp}
\lstinputlisting[language=C++]{../QtGame/Game.cpp}

\setcounter{subsection}{3} 
\textbf{\subsection{  - GameItem.h}}
\addcontentsline{toc}{subsection}{GameItem.h}
\begin{lstlisting}
	QMutex mutex;
public:
int X(){
	mutex.lock();
	auto temp = _x;
	mutex.unlock();
	return temp;
}
int Y(){
	mutex.lock();
	auto temp = _y;
	mutex.unlock();
	return temp;
}
void X( int x ){
	mutex.lock();
	_x = x;
	mutex.unlock();
}
void Y( int y ){
	mutex.lock();
	_y = y;
	mutex.unlock();
}
\end{lstlisting}

\setcounter{subsection}{4} 
\textbf{\subsection{  - MainWindow.h}}
\addcontentsline{toc}{subsection}{MainWindow.h}
\lstinputlisting[language=C++]{../QtGame/MainWindow.h}

\setcounter{subsection}{4} 
\textbf{\subsection{  - MainWindow.cpp}}
\addcontentsline{toc}{subsection}{MainWindow.cpp}
\lstinputlisting[language=C++]{../QtGame/MainWindow.cpp}

\setcounter{subsection}{5} 
\textbf{\subsection{  - Main.cpp}}
\addcontentsline{toc}{subsection}{main.cpp}
\lstinputlisting[language=C++]{../QtGame/main.cpp}

\setcounter{subsection}{6} 
\textbf{\subsection{  - Level1.txt}}
\addcontentsline{toc}{subsection}{level1.txt}
\lstinputlisting[language=C++]{../QtGame/level1.txt}

\setcounter{subsection}{7} 
\textbf{\subsection{  - Level2.txt}}
\addcontentsline{toc}{subsection}{level2.txt}
\lstinputlisting[language=C++]{../QtGame/level2.txt}
\end{document} % конец документа
