%!TEX TS-program = xelatex

% Шаблон документа LaTeX создан в 2018 году
% Алексеем Подчезерцевым
% В качестве исходных использованы шаблоны
% 	Данилом Фёдоровых (danil@fedorovykh.ru) 
%		https://www.writelatex.com/coursera/latex/5.2.2
%	LaTeX-шаблон для русской кандидатской диссертации и её автореферата.
%		https://github.com/AndreyAkinshin/Russian-Phd-LaTeX-Dissertation-Template

\documentclass[a4paper,14pt]{article}

\input{data/preambular.tex}
\begin{document} % конец преамбулы, начало документа
\begin{titlepage}
	\begin{center}
		ФЕДЕРАЛЬНОЕ  ГОСУДАРСТВЕННОЕ АВТОНОМНОЕ \\
		ОБРАЗОВАТЕЛЬНОЕ УЧРЕЖДЕНИЕ ВЫСШЕГО ОБРАЗОВАНИЯ\\
		«НАЦИОНАЛЬНЫЙ ИССЛЕДОВАТЕЛЬСКИЙ УНИВЕРСИТЕТ\\
		«ВЫСШАЯ ШКОЛА ЭКОНОМИКИ»
	\end{center}
	
	\begin{center}
		\textbf{Московский институт электроники и математики}
		
		\textbf{Им. А.Н.Тихонова НИУ ВШЭ}
	\end{center}
	\vspace{1ex}	
	\begin{center}
		Солодянкин Андрей Александрович, группа БИВ172\\
		Подчезерцев Алексей Евгеньевич, группа БИВ172
	\end{center}	
	\vspace{1ex}
	\begin{center}
		\textbf{ОТЧЕТ\\
			ПО ЛАБОРАТОРНОЙ РАБОТЕ 3
		}
	\end{center}	
	\vspace{2ex}
	\begin{center}
		по дисциплине «Электроника»\\
		Тема: «Переходные процессы в RC- и RL- цепях»
	\end{center}
	\vspace{2ex}
	\begin{center}
		Вариант 1.\\
		Дата сдачи отчета: \today
	\end{center}
	\vspace{2ex}
	\vfill
	\begin{center}
		Москва \the\year г.
	\end{center}
\end{titlepage}
\setcounter{page}{2} % нумерация

\renewcommand\contentsname{\centering {\normalsize Содержание}}
\tableofcontents
\newpage

\section*{Постановка задачи}
\addcontentsline{toc}{section}{Постановка задачи}

Рассмотреть паттерн проектирования наблюдатель. Реализовать обработку нажатий клавиш клавиатуры с использованием механизма сигналов и слотов библиотеки \textit{Qt}. Доработать и реализовать логику игры (при нажатии клавиш объекты должны изменять положение).


\newpage

\section{Основная часть}
\subsection{Общая идея решения задачи}
В работе была реализована обработка движения пчелы при нажатии на клавиатуру с помощью слотов \textit{keyReleaseEvent} и \textit{keyPressEvent}. Была улучшена логика игры: появилась возможность убивать врагов, стрелять в облака и ловить колокольчик, выпадающий из облака.
Также для более удобной разработки игры дополнительно были применены паттерны (одиночка, адаптер) и механизм таймера из последующих лабораторных работ.
В проекте был применен поведенческий паттерн проектирования наблюдатель, который позволяет получать объектам оповещения информацию о состоянии уведомителя (таймера).

\subsection{Структура и принципы действия}
Класс \textit{MainWindow} содержит слоты \textit{keyReleaseEvent} и \textit{keyPressEvent}, которые позволяют обрабатывать нажатия с клавиатуры. При нажатии на кнопки влево, вправо, вверх и вниз (обрабатывает слот \textit{keyPressEvent}) пчела перемещается по полю в соответствии с заданным направлением. При нажатии на пробел создается экземпляр класса пули, которая летит вверх. Когда кнопка отпущена, в слоте \textit{keyReleaseEvent} выполняется остановка движения. 

В классе \textit{Game} была реализована функция \textit{Collide}, обрабатывающая столкновения объектов. Отдельно рассмотрено столкновение пули и врага (враг и пуля исчезают, игроку начисляются баллы), столкновение пули и облака (если из облака ранее не вылетал колокольчик, то создается экземпляр класса Bell, пуля исчезает), столкновение пули и колокольчика (изменяется тип движения колокольчика, он поделает вверх, пуля исчезает), столкновение пчелы и колокольчика (колокольчик исчезает, игроку начисляются очки).

В работе был реализован паттерн наблюдатель с использованием механизма таймера (класс \textit{QTimer}). Класс \textit{Notifer} - класс одиночка, который определяет методы подписки, отписки и уведомления наблюдателей. Класс наследуется от класса \textit{QObject} и содержит макрос \textit{Q\_OBJECT}. \textit{Notifer} включает в себя вектор подписчиков, поле \textit{QTimer} (дополнительно фазу и период таймера, период можно задать в \textit{INI}-файле). Содержит слот \textit{Notify}, который уведомляет все объекты из вектора подписчиков. Класс \textit{Observer} является чисто виртуальным классом-подписчиком, который может получать уведомления от \textit{Notifer}. Он содержит единственный метод \textit{Update} для конкретного уведомителя. От класса \textit{Observer} наследуются классы \textit{MainWindow} (чтобы обновлять форму в соответствии с таймером), \textit{GameItem} (чтобы пермещать объекты) и \textit{Game} (для проверки столкновений).

В игру был добавлен класс \textit{Bell}, который приблизит функционал игры к прототипу (класс описывает 2 типа движения в отличие от базового класса \textit{FlyingObj}). 

Был создан класс-адаптер \textit{BulletGen} (также и одиночка), в котором был реализован механизм пистолета. Класс создает экземпляр пули, если текущее состояние кнопки - нажата, а предыдущее - отпущена, и интервал между выстрелами не должен быть меньше \textit{shootTime}. Этот класс работает одновременно и с классом \textit{MainWindow}, и с классом \textit{FlyingObj}. Адаптер получает вызовы от \textit{MainWindow} (клиента), а затем в правильном формате создает экземпляр класса \textit{FlyingObj} (сервиса).

\begin{figure}[H]
	\centering
	\includegraphics[width=\linewidth]{ClassDiagram1.png}
	\caption{Диграма классов}	
\end{figure}
\begin{figure}[H]
	\centering
	\includegraphics[width=\linewidth]{ClassDiagram2.png}
	\caption{Диграма классов (продолжение)}	
\end{figure}

\subsection{Процедура получения исполняемых программных модулей}
Программный код был скомпилирован с среде \textit{Qt Creator}. Компиляция раздельная: исходный код программы разделён на несколько файлов. Никаких дополнительных ключей не добавлялось, использовались ключи, которые добавляются по умолчанию. Параметры сборки: компилятор С++ MinGW 4.8.2, профиль Qt: Qt 4.8.7, отладчик GDB.
\subsection{Результаты тестирования}
Тестирование программы представлено в файле \textit{"main.cpp"} в функции \textit{Main()}. Ожидаемая отрисовка объектов в окне \textit{MainWindow} (убито 2 врага, получен один колокольчик):

\begin{figure}[H]
	\centering
	\includegraphics[width=\linewidth]{test.jpg}
	\caption{Результат тестирования}	
\end{figure}

\newpage
\setcounter{figure}{1} 
\setcounter{section}{1} 
\setcounter{subsection}{1} 

\begin{center}
	\section*{Приложение А}
	полный код программы
	\addcontentsline{toc}{section}{Приложение А}
\end{center}

\renewcommand{\subsection}{\Asbuk{section}.\arabic{subsection}}
\setcounter{subsection}{1} 
\textbf{\subsection{  - Game.h}}
\addcontentsline{toc}{subsection}{Game.h}
\lstinputlisting[language=C++]{../QtGame/Game.h}

\setcounter{subsection}{2} 
\textbf{\subsection{  - GameItem.h}}
\addcontentsline{toc}{subsection}{GameItem.h}
\lstinputlisting[language=C++]{../QtGame/GameItem.h}

\setcounter{subsection}{3} 
\textbf{\subsection{  - Bee.h}}
\addcontentsline{toc}{subsection}{Bee.h}
\lstinputlisting[language=C++]{../QtGame/Bee.h}

\setcounter{subsection}{4} 
\textbf{\subsection{  - Cloud.h}}
\addcontentsline{toc}{subsection}{Cloud.h}
\lstinputlisting[language=C++]{../QtGame/Cloud.h}

\setcounter{subsection}{5} 
\textbf{\subsection{  - FlyingObj.h}}
\addcontentsline{toc}{subsection}{FlyingObj.h}
\lstinputlisting[language=C++]{../QtGame/FlyingObj.h}

\setcounter{subsection}{6} 
\textbf{\subsection{  - Bell.h}}
\addcontentsline{toc}{subsection}{Bell.h}
\lstinputlisting[language=C++]{../QtGame/Bell.h}

\setcounter{subsection}{7} 
\textbf{\subsection{  - Bulletgen.h}}
\addcontentsline{toc}{subsection}{Bulletgen.h}
\lstinputlisting[language=C++]{../QtGame/Bulletgen.h}

\setcounter{subsection}{8} 
\textbf{\subsection{  - Enemy.h}}
\addcontentsline{toc}{subsection}{Enemy.h}
\lstinputlisting[language=C++]{../QtGame/Enemy.h}

\setcounter{subsection}{9} 
\textbf{\subsection{  - Initialization.h}}
\addcontentsline{toc}{subsection}{Initialization.h}
\lstinputlisting[language=C++]{../QtGame/Initialization.h}

\setcounter{subsection}{10} 
\textbf{\subsection{  - MainWindow.h}}
\addcontentsline{toc}{subsection}{MainWindow.h}
\lstinputlisting[language=C++]{../QtGame/mainwindow.h}

\setcounter{subsection}{11} 
\textbf{\subsection{  - Observer.h}}
\addcontentsline{toc}{subsection}{Observer.h}
\lstinputlisting[language=C++]{../QtGame/Observer.h}

\setcounter{subsection}{12} 
\textbf{\subsection{  - GameItem.cpp}}
\addcontentsline{toc}{subsection}{GameItem.cpp}
\lstinputlisting[language=C++]{../QtGame/GameItem.cpp}

\setcounter{subsection}{13} 
\textbf{\subsection{  - Bee.cpp}}
\addcontentsline{toc}{subsection}{Bee.cpp}
\lstinputlisting[language=C++]{../QtGame/Bee.cpp}

\setcounter{subsection}{14} 
\textbf{\subsection{  - Bell.cpp}}
\addcontentsline{toc}{subsection}{Bell.cpp}
\lstinputlisting[language=C++]{../QtGame/Bell.cpp}

\setcounter{subsection}{15} 
\textbf{\subsection{  - Cloud.cpp}}
\addcontentsline{toc}{subsection}{Cloud.cpp}
\lstinputlisting[language=C++]{../QtGame/Cloud.cpp}

\setcounter{subsection}{16} 
\textbf{\subsection{  - Flyingobj.cpp}}
\addcontentsline{toc}{subsection}{Flyingobj.cpp}
\lstinputlisting[language=C++]{../QtGame/Flyingobj.cpp}

\setcounter{subsection}{17} 
\textbf{\subsection{  - Observer.cpp}}
\addcontentsline{toc}{subsection}{Observer.cpp}
\lstinputlisting[language=C++]{../QtGame/Observer.cpp}

\setcounter{subsection}{18} 
\textbf{\subsection{  - main.cpp}}
\addcontentsline{toc}{subsection}{main.cpp}
\lstinputlisting[language=C++]{../QtGame/main.cpp}

\setcounter{subsection}{19} 
\textbf{\subsection{  - MainWindow.cpp}}
\addcontentsline{toc}{subsection}{MainWindow.cpp}
\lstinputlisting[language=C++]{../QtGame/mainwindow.cpp}

\setcounter{subsection}{20} 
\textbf{\subsection{  - Settings.ini}}
\addcontentsline{toc}{subsection}{Settings.ini}
\lstinputlisting{settings.ini}
\end{document} % конец документа
